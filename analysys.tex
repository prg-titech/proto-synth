\documentclass{article}
\usepackage{listings}
\lstdefinelanguage{JavaScript}{
  keywords={break, case, catch, continue, debugger, default, delete, do, else, finally, for, function, if, in, instanceof, new, return, switch, this, throw, try, typeof, var, void, while, with},
  basicstyle = \ttfamily,
  sensitive=true,
  comment=[l]{//},
  morecomment=[s]{/*}{*/},
  morestring=[b]", 
  morestring=[b]'
}
\usepackage{color}
\usepackage{graphicx}

\title{RefSynの手動解析記録}
\author{}
\date{}

\begin{document}

\maketitle

\section{対象関数: \texttt{append}}
現在のノードを先頭とする連結リストの末尾に、新しいノードを追加するメソッド。
引数として渡された値を \texttt{val} フィールドにもつノードを新たに作成し、それをリストの最後に接続。

\subsection{初期プログラム}

\begin{lstlisting}[language=JavaScript]
class Node {
  append(arg) { /* TBD */ }
}
var lst = new Node(); 
lst.val = 2;
lst.append(0);
lst.append(3);
\end{lstlisting}

\subsection{操作ログ}

\subsubsection{\texttt{lst.append(0)} に対する操作}
\begin{itemize}
  \item \texttt{addNode}: \_\_temp1, Node, false, undefined
  \item \texttt{addNode}: \_\_temp2, 0, true, string
  \item \texttt{addEdge}: \_\_temp1, \_\_temp2, val
  \item \texttt{addEdge}: main-new1, \_\_temp1, next
\end{itemize}

\subsubsection{\texttt{lst.append(3)} に対する操作}
\begin{itemize}
  \item \texttt{addNode}: \_\_temp3, Node, false, undefined
  \item \texttt{addNode}: \_\_temp4, 3, true, string
  \item \texttt{addEdge}: \_\_temp3, \_\_temp4, val
  \item \texttt{addEdge}: temp1, \_\_temp3, next
\end{itemize}

\subsection{手動で復元したコード}

\subsubsection{\texttt{lst.append(0)}}
\begin{lstlisting}[language=JavaScript]
var temp1 = new Node();
var temp2 = 0;
temp1.val = temp2;
this.next = temp1;
\end{lstlisting}

\subsubsection{\texttt{lst.append(3)}}
\begin{lstlisting}[language=JavaScript]
var temp1 = new Node();
var temp2 = 3;
temp1.val = temp2;
this.next.next = temp1;
\end{lstlisting}

\subsection{一般化と部分関数の推定}

\begin{lstlisting}[language=JavaScript]
var temp1 = new Node();
var temp2 = f();    // f() = 引数 arg を返す関数
temp1.val = temp2;
g().next = temp1;   // g() = 現在のリストの末尾を返す関数
\end{lstlisting}

\subsection{想定の合成された関数}
\begin{lstlisting}
append(arg) {
  var temp1 = new Node();
  function f() {
    return arg;
  }
  var temp2 = f();
  temp1.val = temp2;
  function g() {
    if(!this.next) {
      return this;
    } else {
      this.next.g();    // できないけど
    }
  }
  g().next = temp1;
}
\end{lstlisting}

\subsection{oracleの関数}
\begin{lstlisting}[language=JavaScript]
append(arg) {
  if(!this.next) {
    this.next = new Node();
    this.net.val = arg;
  } else {
    this.next.append(arg);
  }
}
\end{lstlisting}


% \clearpage

%====================================================================================
\section{対象関数: \texttt{prepend}}
現在のノードを先頭とする連結リストの先頭に、新しいノードを追加するメソッド。
引数として渡された値を \texttt{val} フィールドにもつノードを新たに作成し、それをリストの先頭に接続。

\subsection{初期プログラム}

\begin{lstlisting}[language=JavaScript]
class Node {
  prepend(arg) { /* TBD */ }
}
var lst = new Node(); 
lst.val = 2;
lst = lst.prepend(0);
lst = lst.prepend(3);
\end{lstlisting}

\subsection{操作ログ}

\subsubsection{\texttt{lst.prepend(0)} に対する操作}
\begin{itemize}
  \item \texttt{addNode}: \_\_temp1, Node, false, undefined
  \item \texttt{addNode}: \_\_temp2, 0, true, string
  \item \texttt{addEdge}: \_\_temp1, \_\_temp2, val
  \item \texttt{addEdge}: \_\_temp1, main-new1, next
  \item \texttt{addVariable}: \_\_temp1, return
\end{itemize}

\subsubsection{\texttt{lst.prepend(3)} に対する操作}
\begin{itemize}
  \item \texttt{addNode}: \_\_temp3, Node, false, undefined
  \item \texttt{addNode}: \_\_temp4, 3, true, string
  \item \texttt{addEdge}: \_\_temp3, \_\_temp4, val
  \item \texttt{addEdge}: \_\_temp3, \_\_temp1, next
  \item \texttt{addVariable}: \_\_temp3, return
\end{itemize}

\subsection{手動で復元したコード}

\subsubsection{\texttt{lst.prepend(0)}}
\begin{lstlisting}[language=JavaScript]
var temp1 = new Node();
var temp2 = 0;
temp1.val = temp2;
temp1.next = this;
return temp1;
\end{lstlisting}

\subsubsection{\texttt{lst.prepend(3)}}
\begin{lstlisting}[language=JavaScript]
var temp3 = new Node();
var temp4 = 3;
temp3.val = temp4;
temp3.next = this;
return temp3;
\end{lstlisting}

temp1をthisにすることは実行時情報から判断できる?

\subsection{一般化と部分関数の推定}

\begin{lstlisting}[language=JavaScript]
var temp1 = new Node();
var temp2 = f();    // f() = 引数 arg を返す関数
temp1.val = temp2;
temp1.next = this;
return temp1;
\end{lstlisting}

\subsection{想定の合成された関数}
\begin{lstlisting}
prepend(arg) {
  var temp1 = new Node();
  function f() {
    return arg;
  }
  var temp2 = f();
  temp1.val = temp2;
  temp1.next = this;
  return temp1;
}
\end{lstlisting}

\subsection{oracleの関数}
\begin{lstlisting}[language=JavaScript]
prepend(arg) {
  var temp1 = new Node();
  temp1.val = arg;
  temp1.next = this;
  return temp1;
}
\end{lstlisting}


% \clearpage


%====================================================================================
\section{対象関数: \texttt{removeLast}}
現在のノードを先頭とする連結リストの最後尾のノードを消すメソッド。

\subsection{初期プログラム}

\begin{lstlisting}[language=JavaScript]
class Node {
    removeLast() { /* TBD */ }
}
var lst = new Node(); 
lst.val = 2;
lst.next = new Node();
lst.next.val = 0;
lst.next.next = new Node();
lst.next.next.val = 3;
lst.removeLast();
lst.removeLast();
\end{lstlisting}

\subsection{操作ログ}

\subsubsection{\texttt{lst.removeLast()} に対する操作}
\begin{itemize}
  \item \texttt{deleteNode}: main-new3
\end{itemize}

\subsubsection{\texttt{lst.removeLasr()} に対する操作}
\begin{itemize}
  \item \texttt{deleteNode}: main-new2
\end{itemize}

nextとかvalとかも消しても良い  \\
Nodeだけ消せば可視化されない

\subsection{手動で復元したコード}

\subsubsection{\texttt{lst.removeLast()}}
\begin{lstlisting}[language=JavaScript]
this.next.next = undefined
\end{lstlisting}

\subsubsection{\texttt{lst.removeLast()}}
\begin{lstlisting}[language=JavaScript]
this.next = undefined
\end{lstlisting}

next先のNodeが消えた  \\
nullではなく、undefined  \\
constructorがあるならnullでよい 

\subsection{一般化と部分関数の推定}

\begin{lstlisting}[language=JavaScript]
f().next = undefined;   // f() = 現在のリストの末尾を返す関数
\end{lstlisting}

\subsection{想定の合成された関数}
\begin{lstlisting}
removeLast() {
  function f() {
    if(!this.next.next) {
      return this.next;
    } else {
      this.next.g();    // できないけど
    }
  }
  f().next = undefined;
}
\end{lstlisting}

\subsection{oracleの関数}
\begin{lstlisting}[language=JavaScript]
removeLast() {
  if(this.next.next === undefined) {
    this.next = undefined;
  } else {
    this.next.removeLast();
  }
}
\end{lstlisting}

% \clearpage





%====================================================================================
\section{対象関数: \texttt{removeFirst}}
現在のノードを先頭とする連結リストの先頭のノードを消すメソッド。

\subsection{初期プログラム}

\begin{lstlisting}[language=JavaScript]
class Node {
    removeFirst() { /* TBD */ }
}
var lst = new Node(); 
lst.val = 2;
lst.next = new Node();
lst.next.val = 0;
lst.next.next = new Node();
lst.next.next.val = 3;
lst = lst.removeFirst();
lst = lst.removeFirst();
\end{lstlisting}

\subsection{操作ログ}

\subsubsection{\texttt{lst.removeFirst()} に対する操作}
\begin{itemize}
  \item \texttt{addVariable}: main-new2, return
\end{itemize}

\subsubsection{\texttt{lst.removeFirst()} に対する操作}
\begin{itemize}
  \item \texttt{addVariable}: main-new3, return
\end{itemize}

\subsection{手動で復元したコード}

\subsubsection{\texttt{lst.prepend(0)}}
\begin{lstlisting}[language=JavaScript]
return this.next;
\end{lstlisting}
ここの変換は非自明?


\subsubsection{\texttt{lst.prepend(3)}}
\begin{lstlisting}[language=JavaScript]
return this.next;
\end{lstlisting}

\subsection{一般化と部分関数の推定}

\begin{lstlisting}[language=JavaScript]
return this.next;
\end{lstlisting}

\subsection{想定の合成された関数}
\begin{lstlisting}
removeFirst() {
  return this.next;
}
\end{lstlisting}

\subsection{oracleの関数}
\begin{lstlisting}[language=JavaScript]
removeFirst() {
  return this.next;
}
\end{lstlisting}

% \clearpage





%====================================================================================



%====================================================================================
\section{対象関数: \texttt{removeVal(arg)}}
連結リスト内のvalフィールドがargの値を持つノードを削除するメソッド。

\subsection{初期プログラム}

\begin{lstlisting}[language=JavaScript]
class Node {
  removeVal(arg) { /* TBD */ }
}
var lst = new Node(); 
lst.val = 2;
lst.next = new Node();
lst.next.val = 0;
lst.next.next = new Node(); 
lst.next.next.val = 3;
lst.next.next.next = new Node(); 
lst.next.next.next.val = 1;
lst = lst.removeVal(0);
lst = lst.removeVal(1);
\end{lstlisting}

\subsection{操作ログ}

\subsubsection{\texttt{lst.removeVal(0)} に対する操作}
\begin{itemize}
  \item \texttt{editEdgeReference}: main-new1, main-new2, main-new3, next
  \item \texttt{addVariable}: main-new1, return
\end{itemize}

\subsubsection{\texttt{lst.removeVal(1)} に対する操作}
\begin{itemize}
  \item \texttt{deleteEdge}: main-new3, main-new4, next
  \item \texttt{addVariable}: main-new1, return
\end{itemize}

\subsection{手動で復元したコード}

\subsubsection{\texttt{lst.removeVal(0)}}
\begin{lstlisting}[language=JavaScript]
this.next = this.next.next;
return this;
\end{lstlisting}

\subsubsection{\texttt{lst.removeVal(1)}}
\begin{lstlisting}[language=JavaScript]
this.next.next = null;
return this;
\end{lstlisting}

\subsection{一般化と部分関数の推定}

\begin{lstlisting}[language=JavaScript]
f().next = g()
// f は val フィールドに arg をもつノードの一個前
// g は val フィールドに arg をもつノードの一個後
\end{lstlisting}

\subsection{想定の合成された関数}
\begin{lstlisting}
removeVal(arg) {
  f().next = g();
  return this;
  f() {
    if(this.val === arg) {
      
    }
  }
  g() {
    if(this.val === arg) {
    
    }
  }
}
\end{lstlisting}

\subsection{oracleの関数}
\begin{lstlisting}[language=JavaScript]
removeVal(arg) {
  if(this.val === arg) {
    return this.next;
  }
  if(this.next !== null) {
    this.next = this.next.removeval(arg);
  }
  return this;
}
\end{lstlisting}


% \clearpage




%====================================================================================

%====================================================================================

%====================================================================================

%====================================================================================

%====================================================================================

%====================================================================================

%====================================================================================

%====================================================================================

%====================================================================================

%====================================================================================

%====================================================================================

%====================================================================================

%====================================================================================


%====================================================================================





\section{対象関数: \texttt{removeAt(i)}}
連結リストのi番目のノードを削除するメソッド。

\subsection{初期プログラム}

\begin{lstlisting}[language=JavaScript]
class Node {
  removeAt(i) { /* TBD */ }
}
var lst = new Node(); 
lst.val = 2;
lst.next = new Node();
lst.next.val = 0;
lst.next.next = new Node(); 
lst.next.next.val = 3;
lst.next.next.next = new Node(); 
lst.next.next.next.val = 1;
lst.removeAt(2);
lst.removeAt(1);
\end{lstlisting}

\subsection{操作ログ}

\subsubsection{\texttt{lst.removeAt(2)} に対する操作}
\begin{itemize}
  \item \texttt{editEdgeReference}: main-new2, main-new3, main-new4, next
  \item \texttt{addVariable}: main-new1, return
\end{itemize}

\subsubsection{\texttt{lst.removeAt(1)} に対する操作}
\begin{itemize}
  \item \texttt{editEdgeReference}: main-new1, main-new2, main-new4, next
  \item \texttt{addVariable}: main-new1, return
\end{itemize}

\subsection{手動で復元したコード}

\subsubsection{\texttt{lst.append(0)}}
\begin{lstlisting}[language=JavaScript]
this.next.next = this.next.next.next;
return this;
\end{lstlisting}

\subsubsection{\texttt{lst.append(3)}}
\begin{lstlisting}[language=JavaScript]
this.next = this.next.next;
return this;
\end{lstlisting}

\subsection{一般化と部分関数の推定}

\begin{lstlisting}[language=JavaScript]
f().next = g(); 
\\ f() = i-1 番目のノードを返す関数
\\ g() = i+1 番目のノードを返す関数
return this;
\end{lstlisting}

\subsection{想定の合成された関数}
\begin{lstlisting}
removeAt(i) {
  f().next = g();
  return this;
  \\ f() = i-1 番目のノードを返す関数
  \\ g() = i+1 番目のノードを返す関数
}
\end{lstlisting}

\subsection{oracleの関数}
\begin{lstlisting}[language=JavaScript]
removeAt(i) {
  if(i=== 0) {
    return this.next;
  }
  this.next = this.next.remove(i - 1);
  return this;
}
\end{lstlisting}


% \clearpage






\section{対象関数: \texttt{insertAfter(i, arg)}}
連結リストのi番目の後ろに、valフィールドにargを持つ新しいノードを追加するメソッド。


\subsection{初期プログラム}

\begin{lstlisting}[language=JavaScript]
class Node {
  insertAfter(i, arg) { /* TBD */ }
}
var lst = new Node(); 
lst.val = 2;
lst.next = new Node();
lst.next.val = 0;
lst.insertAfter(0,3);
lst.insertAfter(1,4);
\end{lstlisting}

\subsection{操作ログ}

\subsubsection{\texttt{lst.insertAfter(0,3)} に対する操作}
\begin{itemize}
  \item \texttt{addNode}: \_\_temp1, Node, false, undefined
  \item \texttt{addNode}: \_\_temp2, 3, true, string
  \item \texttt{addEdge}: \_\_temp1, \_\_temp2, val
  \item \texttt{addEdge}: \_\_temp1, main-new2, next
  \item \texttt{editEdgeReference}: main-new1, main-new2, \_\_temp1, next
\end{itemize}

\subsubsection{\texttt{lst.insertAftere(1,4)} に対する操作}
\begin{itemize}
  \item \texttt{addNode}: \_\_temp3, Node, false, undefined
  \item \texttt{addNode}: \_\_temp4, 4, true, string
  \item \texttt{addEdge}: \_\_temp3, \_\_temp4, val
  \item \texttt{addEdge}: \_\_temp3, main-new2, next
  \item \texttt{editEdgeReference}: \_\_temp1, main-new2, \_\_temp3, next
\end{itemize}

\subsection{手動で復元したコード}

\subsubsection{\texttt{lst.insertAfter(0,3)}}
\begin{lstlisting}[language=JavaScript]
var temp1 = new Node();
var temp2 = 3;
temp1.val = temp2;
temp1.next = this.next;
this.next = temp1;
\end{lstlisting}

\subsubsection{\texttt{lst.insetAfter(1,4)}}
\begin{lstlisting}[language=JavaScript]
var temp1 = new Node();
var temp2 = 4;
temp1.val = temp2;
temp1.next = this.next.next;
this.next.next = temp1;
\end{lstlisting}

\subsection{一般化と部分関数の推定}

\begin{lstlisting}[language=JavaScript]
var temp1 = new Node();
var temp2 = f();        // f() = 引数 arg を返す関数
temp1.val = temp2;
temp1.next = g()    // g() = i+1 番目のノードを返す関数
h().next = temp1;   // h() = i-1 番目のノードを返す関数
\end{lstlisting}

\subsection{想定の合成された関数}
\begin{lstlisting}
insertAfter(i,arg) {
  var temp1 = new Node();
  var temp2 = arg;
  temp1.val = temp2;
  temp1.next = find(i+1);
  find(i-1).next = temp1;
}
\end{lstlisting}

\subsection{oracleの関数}
\begin{lstlisting}[language=JavaScript]
insertAfter(i, arg) {
  function find(node, index) {
    if (node === null) {
      return null;
    }
    if (index === 0) {
      return node;
    }
    return find(node.next, index - 1);
  }
  const targetNode = find(this, i);
  if (targetNode !== null) {
    const newNode = new Node();
    newNode.val = arg;
    newNode.next = targetNode.next;
    targetNode.next = newNode;
  }
}
\end{lstlisting}

% \clearpage




\section{対象関数: \texttt{concat(lst)}}
現在のノードを先頭とする連結リストの末尾に、別の連結リストを追加するメソッド。

\subsection{初期プログラム}

\begin{lstlisting}[language=JavaScript]
class Node {
  concat(otherList) { /* TBD */ }
}
var lst = new Node(); 
lst.val = 2;
lst.next = new Node();
lst.next.val = 0;
lst.next.next = new Node(); 
lst.next.next.val = 3;
var list = new Node(); 
list.val = 1;
list.next = new Node();
list.next.val = 4;
var l = new Node();
l.val = 5;
lst.concat(list);
lst.concat(l);
\end{lstlisting}

\subsection{操作ログ}

\subsubsection{\texttt{lst.concat(list)} に対する操作}
\begin{itemize}
  \item \texttt{addEdge}: main-new3, main-new4, next
\end{itemize}

\subsubsection{\texttt{lst.concat(l)} に対する操作}
\begin{itemize}
  \item \texttt{addNode}: main-new5, main-new6, next
\end{itemize}

\subsection{手動で復元したコード}

\subsubsection{\texttt{lst.concat(list)}}
\begin{lstlisting}[language=JavaScript]
this.next.next = list;
\end{lstlisting}

\subsubsection{\texttt{lst.concat(l)}}
\begin{lstlisting}[language=JavaScript]
this.next.next.next.next.next = l;
\end{lstlisting}

\subsection{一般化と部分関数の推定}

\begin{lstlisting}[language=JavaScript]
f().next = g();
\\ f() = 最後のノードを返す関数
\\ g() = 引数を返す関数
\end{lstlisting}

\subsection{想定の合成された関数}
\begin{lstlisting}
concat(otherList) {
  f().next = otherList;
  \\ f() = thisの最後のノードを返す関数
}
\end{lstlisting}

\subsection{oracleの関数}
\begin{lstlisting}[language=JavaScript]
concat(otherList) {
  if(!this.next) {
    this.next = otherList;
  } else {
    this.next.concat(otherList);
  }
}
\end{lstlisting}


% \clearpage




\section{対象関数: \texttt{set(i,arg)}}
連結リストのi番目の値をargに変更するメソッド。

\subsection{初期プログラム}

\begin{lstlisting}[language=JavaScript]
class Node {
  set(i,arg) { /* TBD */ }
}
var lst = new Node(); 
lst.val = 2;
lst.next = new Node();
lst.next.val = 0;
lst.next.next = new Node(); 
lst.next.next.val = 3;
lst.next.next.next = new Node(); 
lst.next.next.next.val = 1;
lst.set(2,5);
lst.set(1,4);
\end{lstlisting}

\subsection{操作ログ}

\subsubsection{\texttt{lst.set(2,5)} に対する操作}
\begin{itemize}
  \item \texttt{addNode}: \_\_temp1, 5, true, string
  \item \texttt{editEdgeReference}: main-new3, main-new3-val, \_\_temp1, val
\end{itemize}

\subsubsection{\texttt{lst.set(1,4)} に対する操作}
\begin{itemize}
  \item \texttt{addNode}: \_\_temp2, 4, true, string
  \item \texttt{addEdge}: main-new2, main-new2-val, \_\_temp2, val
\end{itemize}

\subsection{手動で復元したコード}

\subsubsection{\texttt{lst.set(2,5)}}
\begin{lstlisting}[language=JavaScript]
var temp1 = 5;
this.next.next.val = temp1;
\end{lstlisting}

\subsubsection{\texttt{lst.set(1,4)}}
\begin{lstlisting}[language=JavaScript]
var temp1 = 4;
this.next.val = temp2;
\end{lstlisting}

\subsection{一般化と部分関数の推定}

\begin{lstlisting}[language=JavaScript]
var temp1 = f();    \\ f() = 引数 arg を返す関数
g().val = temp1;    \\ g() = i番目のノードを返す関数
\end{lstlisting}

\subsection{想定の合成された関数}
\begin{lstlisting}
set(i,arg) {
  var temp1 = arg;
  find(i).val = temp1;
}
\end{lstlisting}

\subsection{oracleの関数}
\begin{lstlisting}[language=JavaScript]
set(i,arg) {
  if(i === 0) {
    this.val = arg;
  }
  if(i > 0) {
    this.next.set(i-1,arg);
  }
}
\end{lstlisting}


% \clearpage


\section{対象関数: \texttt{swap(i, j)}}
現在のノードを先頭とする連結リストのi番目とj番目の要素を交換するメソッド。

\subsection{初期プログラム}

\begin{lstlisting}[language=JavaScript]
class Node {
  swap(i,j) { /* TBD */ }
}
var lst = new Node(); 
lst.val = 2;
lst.next = new Node();
lst.next.val = 0;
lst.next.next = new Node(); 
lst.next.next.val = 3;
lst.next.next.next = new Node(); 
lst.next.next.next.val = 1;
lst.swap(0, 3);
lst.swap(1,2);
\end{lstlisting}

\subsection{操作ログ}

\subsubsection{\texttt{lst.swap(0,3)} に対する操作}
\begin{itemize}
  \item \texttt{addNode}: \_\_temp1, 2, true, string
  \item \texttt{addNode}: \_\_temp2, 1, true, string
  \item \texttt{editEdgeReference}: main-new1, main-new1-val, \_\_temp1, val
  \item \texttt{editEdgeReference}: main-new4, main-new4-val, \_\_temp2, val
\end{itemize}

\subsubsection{\texttt{lst.swap(1,2)} に対する操作}
\begin{itemize}
  \item \texttt{addNode}: \_\_temp3, 3, true, string
  \item \texttt{addNode}: \_\_temp4, 0, true, string
  \item \texttt{editEdgeReference}: main-new2, main-new2-val, \_\_temp3, val
  \item \texttt{editEdgeReference}: main-new3, main-new3-val, \_\_temp4, val
\end{itemize}

\subsection{手動で復元したコード}
操作列だけみると実際には違う挙動をするプログラムが出てくる
参照元がなくなったが使うものは先にtempとかで変数宣言しておく?
\subsubsection{\texttt{lst.swap(0,3)}}
\begin{lstlisting}[language=JavaScript]
var temp1 = 2;
var temp2 = 1;
this.val = temp1;
this.next.next.next.val = temp2;
\end{lstlisting}

\subsubsection{\texttt{lst.swap(1,2)}}
\begin{lstlisting}[language=JavaScript]
var temp1 = 3;
var temp2 = 0;
this.next.val = temp1;
this.next.next.val = temp2;
\end{lstlisting}

\subsection{一般化と部分関数の推定}
\begin{lstlisting}[language=JavaScript]
var temp1 = f();    \\ f() = i 番目のノードの val フィールドの値を返す関数
var temp2 = g();    \\ g() = j 番目のノードの val フィールドの値を返す関数
h().val = temp1;    \\ h() = j 番目のノードを返す関数
i().val = temp2;    \\ i() = i 番目のノードを返す関数
\end{lstlisting}

\subsection{想定の合成された関数}
\begin{lstlisting}
swap(i,j) {
  var temp1 = getVal(i);
  var temp2 = getVal(j);
  getNode(j).val = temp1;
  getNode(i).val = temp2;
}
\end{lstlisting}

\subsection{oracleの関数}
\begin{lstlisting}[language=JavaScript]
swap(i,j) {
  function find(node, index) {
    if(index === 0) {
      return node;
    }
    return find(node.next, index - 1);
  }
  var nodeI = find(this, i);
  var nodeJ = find(this, j);
  var tempVal = nodeI.val;
  nodeI.val = nodeJ.val;
  nodeJ.val = tempVal;
}
\end{lstlisting}


% \clearpage




\end{document}
